\documentclass[12pt,oneside,a4paper]{article}
%
\usepackage{graphicx}
\usepackage{tabularx}
\usepackage{url}
\usepackage[latin1]{inputenc}
%\usepackage[utf8]{inputenc}
\usepackage[left]{eurosym} 
\usepackage{times}
\usepackage{pdfpages}
\usepackage{todonotes}
\usepackage{enumerate}

% We don't want a special font for urls (looks bad with times):
\urlstyle{same}

% Graphics extensions and path:
%\DeclareGraphicsExtensions{.pdf,.png,.jpg}
%\DeclareGraphicsExtensions{.eps,.ps}
%\graphicspath{{figures/}}

%
% Page size
%
\usepackage[top=25mm,left=25mm,right=25mm,bottom=25mm]{geometry}

\setlength{\headheight}{15pt}    % Necessary to avoid fancyhdr warning.


\newcommand\LongTitle{Humidity Retrievals over Arctic}
\newcommand\ShortTitle{Arctic Humidity}



%
% Heading
%
\newcommand{\pagevers}[2]{
\ifnum\thepage=1 
#1
\else#2
\fi
}
%
\usepackage{fancyhdr}
\pagestyle{fancy}
\chead{}
%\rhead{\pagevers{}{\bf \thepage}}
\rhead{\thepage}
\rfoot{\small \it \ShortTitle\ ---\  Application to SNSB 2021-R}
\cfoot{}
\lfoot{}
%\renewcommand{\headrulewidth}{\pagevers{0pt}{0.2pt}}
\renewcommand{\headrulewidth}{0.2pt}
\renewcommand{\footrulewidth}{0pt}


\newcommand{\docname}[1]{\lhead{\small #1}}

%
% Section titles
%
\usepackage[small]{titlesec}
\titlespacing*{\section}{0pt}{*3.3}{*0.5}
%
% 
%
\def\compactitems{\parskip0pt\topsep0pt\partopsep0pt\parsep0pt\itemsep0pt}

% Struts for better table formatting:
\newcommand\T{\rule{0pt}{2.6ex}}
\newcommand\B{\rule[-1.2ex]{0pt}{0pt}}


\newcommand{\FIXME}[1]{{\bfseries \textcolor{red}{FIXME:} #1}}
\newcommand{\md}[1]{\mbox{#1-d}}
%\newcommand{\3d}{3d}

%\hyphenation{3-d}
\uchyph=0

%%% Local Variables: 
%%% mode: latex
%%% TeX-master: t
%%% End: 



\docname{Project description}
\usepackage{graphicx}
\usepackage{caption}
\usepackage{subcaption}
\captionsetup[figure]{font=normalsize,labelfont=normalsize}

%
% References
%
\usepackage{natbib}
\bibliographystyle{agu04}     
\setlength{\bibsep}{0mm}


\newcommand\wpstart[3]{\noindent\textbf{WP #1, #2}\hspace{\stretch{1}}Priority #3%
	\vspace{-4mm}\\\rule{\textwidth}{0.5pt}\\}
\newcommand\wpenda[4]{%
	\noindent -----\\ 
	\begin{tabularx}{0.95\hsize}{l p{133mm}}    
		\hspace*{-1.1ex}Start\,--\,end: & #1\\
		\hspace*{-1.1ex}Main output: & #2\\
		\hspace*{-1.1ex}Main risks: & #3\\
	\end{tabularx}\\
	\vspace{-2.2ex}\noindent\rule{\textwidth}{0.5pt}\\
}


\begin{document}
	
	
	\thispagestyle{empty}
	\vspace*{-10mm}
	\noindent
	\textbf{\Large \LongTitle}




\section{General summary}

\subsection{Background}
%
\label{sec:background}
Some info from Luisa (1/2 page)

\subsection{Previous works}
%
\label{sec:previousworks}
Accurate measurements of water vapour profile over polar regions is through ground based microwave radiometers or radiosondes. For a comprehensive overview as required for monitoring purposes can only be achieved through space borne observations. However, over polar regions, the main challenge encountered by microwave (MW) observations based retrieval algorithms is the high and highly variable surface emissivity which dominates the signal. The most important work towards retreival of WVP from microwave humidity sounders (such as Advanced Microwave Sounding Unit-B (AMSU-B) and Microwave Humidity Sounder (MHS)) comes from University of Bremen. Their retrieval concept was initiated by \citet{miao:2001:atmos}, where they utilized water vapour absorption channels around 183\,GHz and  150\,GHz window channel to retrieve total water vapour (TWV) upto 7\,kg m$^{-2}$. Subsequently in the study \citep{} they extended this approach to include 89\,GHz to retrieve TWV upto 15\,kg m$^{-2}$ over sea-ice regions and formulated a relationship between sea-ice emissivity over different frequencies using measurement campaigns. Later, \cite{scarlat:2018:retri} extended the method to include all surface types using AMSU-B. A comparison of the retrieved WVP against ERA-Interim showed that the over winter months, the RMSD was  1.86\,kg m$^{-2}$ but over summer months the errors were up to 5.67 m$^{-2}$ due to the algorithm being constrained by its upper retrieval limit of 15 m$^{-2}$. Besides MW sounding, an attempt at TWV has also been made using low frequency microwave observations from Advanced Microwave Scanning Radiometer (AMSR). For example, \citet{scarlat:2017:exper} use optimal estimation (OEM) for multi-parameter retrieval over Arctic, and \citet{zabolotskikh:2020:anadv} attempt at TWV retrieval over both open ocean and sea-ice regions using neural networks based inversion. In both products, the highest uncertainties in the retrieval are linked to the empirical estimates of surface emissivity over sea-ice regions.


The lack of an accurate atmospheric data over the extensive Arctic sea-ice regions has implications in the forecast skill of numerical weather prediction (NWP) models. Though the  models are themselves suffer from limitations associated with the modelling of snow, sea ice, mixed-phase clouds, \citet{lawrence:2019:usean} show that MW sounding observations have a clear positive impact on the predictive skill in the summer months. Over the winter months the optimal utilisation of the observations is constrained by the presence of snow and sea-ice. Infact,  increasing the usage of satellite data over all surfaces in Integrated Forecast System (IFS) is identified as one of the priorities in ECMWF Strategy for 2021-2030. 


\section{Project description}

The retrieval of atmospheric parameters from brightness temperatures is an inverse problem. The Bayesian retrieval methods provide a way to handle the ill-posedness of the retrieval problem and its associated uncertainties. In this project, we utilize  Quantile Regression Neural Network (QRNN), a machine learning technique to invert the simulated brightness temperatures (TB) to WVP. The database with simulated TB profiles is generated using Atmopsheric Radiative Transfer Simulator (ARTS). 
 
\subsection{Tools}
\subsubsection{QRNN}
%
\label{sec:qrnn}
The neural network (NN) training is a process of learning to predict the outputs {$y_i$} from inputs {$x_i$} through a series of learnable transformations. In traditional NN techniques, the output is a point estimate of the target variable. However, QRNN is trained to minimise the mean of the quantile loss function and predict chosen quantiles of its Bayesian a posterior distribution. QRNN can be seen as a machine learning version of Bayesian Monte Carlo integration (BMCI) to solve ill-posed problems. A detailed description of QRNN can be found in \citet{pfreundschuh:aneur:18}.  

In all the applications QRNN has been tested so far, it has outperformed the existing approaches. This includes a very recent study by the main applicant for predicting clear noise-free clear-sky radiances from microwave humidity channels \citep{kaur:2021:canma}. Previously, \citet{pfreundschuh:aneur:18} had shown the advanatge of QRNN in predicting cloud top pressure from observations by the Moderate Resolution Imaging Spectroradiometer (MODIS). Recent studies with QRNN also include working with GPROF team  to replace BMCI by QRNN for GPROF retrievals (manuscript in preparation).


\subsubsection{ARTS}
\label{sec:arts}
% 
The backbone of our work is buliding the database with simulated TB using ARTS (Atmospheric Radiative Transfer Simulator, \url{www.radiativetransfer.org}). ARTS has some unique features, but in this context it is rather the completeness and flexibility of ARTS that is
helpful. There is now a second cornerstone of the ARTS infrastructure, the
associated database of single scattering properties \citep{eriksson:agene:18}.
The main part contains data for 36 particle ``habits'' assuming totally random
orientation (TRO). Already this makes the database the most comprehensive one.
Some data for azimuthally random orientation (ARO) are also at hand
\citep{brath:micro:20,ekelund:micro:20}. In principle more ARO data of ice
hydrometeors are needed, but in \citet{baralakas:intro:21} we show that the ARO
case can be fairly well approximated by scaling TRO data. In an ongoing master
thesis project data for melting particles are being generated, and this then
fills the main remaining gap in the database.

\subsection{Preliminary results}
%
This project will build up on the ongoing efforts, hence we shall briefly summarize the preparatory steps we have undertaken to demonstrate the feasibility of the project. 

\subsubsection{Radiative transfer simulations}
%
Towards creating as realistic and comprehensive retrieval databases as possible, we prepare for the complex radiative transfer calculations using the Global Precipitation Measurement (GPM) Microwave Imager (GMI) frequencies and polarisations. GMI provides observations upto 65$^{\circ}$ and is used as a baseline to formulate the full range of atmospheric and oceanic conditions, including snow and sea-ice emissivities over higher latitudes. The GMI measurements are simulated using Cloudsat radar measurments. A dbZ based systems as described by \citet{ekelund:using:20} is followed and onion peeling retreival method is used to invert Cloudsat reflectivities to Ice Water Content (IWC). The emissivity over land and water are taken from Tool to Estimate Land-Surface Emissivities at Microwave frequencies \citep{aires} and Tool to Estimate Sea-Surface Emissivity from Microwaves to sub-Millimeter waves \citep{prigent}. However for snow and sea-ice surface types, experimental and modelling studies \citep{harlow:2009:milli, harlow:2012:tundr,hewison:2002:airbo} are used to define the valid ranges of emissivity variability for 166\,GHz and 183 \,GHz frequencies. Differences between H and V polaristaion are also taken into account. For each snow/sea-ice observation (identified using ERA5), the corresponding emissivity is set randomly within the ranges of variability.





\subsubsection{Training database}
%
\begin{figure*}[t]
	\centering
	\caption{}
	\begin{subfigure}{.45\textwidth}
		\caption{ Snow}
		\includegraphics[height = 50mm]{Figures/hist2d_gmi_45-60_snow.png}
	\end{subfigure}
	\begin{subfigure}{.45\textwidth}
		\caption{ Sea-ice}
		\includegraphics[height = 50mm]{Figures/hist2d_gmi_highlat_sea-ice.png}
	\end{subfigure}
	\label{fig:histogram_2d}
\end{figure*}

In order to verify that the data in the training database covers the actual measurement space, the measured and simulated TB from GMI are compared. Figure~\ref{fig:histogram_2d} shows the polarisation difference (PD) and TB histograms from 166 GHz for different surface types. The PD is defined as difference between V and H polarisations. Over both sea-ice and snow, the PDs are mostly positive as a consequence of hydrometeor scattering, and the negative PDs arise from noise in clear-sky measurements. For all surface types, the availability of our simulations is higher than the GMI measurements. 
\subsubsection{Retrievals}

\begin{figure*}[t]
	\centering
	\caption{WVP scatter}
	\includegraphics[height = 50mm]{Figures/WVP_scatter_monthlymean.png} 
	\label{fig:wvp_scatter}
\end{figure*}

The QRNN algorithm described in sec~\ref{sec:qrnn} is applied to TB from 166 \,GHz and 183\,GHz to retrieve the corresponding WVPs. A comparison of the monthly means for the retrieved WVP and ERA5 WVP is displayed in fig.~\ref{fig:wvp_scatter}. To enable comparison on common scales, both datasets were re-gridded to 2.5$^{\circ}$. The correlation between the two datasets is 0.97. 


\subsection{Part A : Retrievals}

\subsubsection{Overview}


WPs - Simulate atmospheric scenarios (input from HARMONIE ?)

	- Sea-ice Emissivity 
		- HR identification of sea-ice 	(info from Leif, figure?)
		- 1D VAR retrievals of sea-ice emissivity?  
	

	
	- Database creation (Pencil Beam calculations) +  2d Antenna response, 
	
	- Retrieval Set-up 
	
 
\subsection{Data Analysis}

WPs  - Comparison with existing WV products


\subsection{Application}

WPs  - Assimilation?

\subsection{Risk Assessment}

\section{Collaborations}

\section{Staff situation}
\label{sec:staff}
%
Inderpreet Kaur (IK) started as a postdoctoral researcher in the division during March 2020. She has a background in geophysical parameter retrieval from infra-red frequencies, and worked with various satellite derived geophysical parameters. She is presently working with both millimetre (mm) and sub-millimetre (mm) data for retrieval of atmospheric ice, and is interested in machine learning techniques for retrieval applications. 

Patrick Eriksson (PE) ...

Luisa Ickes (LI) started as an assistant professor during Jan 2020 and is advisor of PhD student Hannah Imhof (HI). Her general field is atmospheric modelling and the special research interests are the Arctic region and clouds in general. She will provide expertise in cloud physics, both when generating training data and evaluating retrievals.

Leif Eriksson (LE) is group leader for the research group Radar Remote Sensing. His research is aimed at development of methods to make measurements of land and ocean with radar data. In this project, his expertise is sought on the use of synthetic aperture radar (SAR) observations to identify sea ice concentration.

With respect to gender aspects, it is pointed out that the last three recruitments to this unit in formation are females (LI, IK and  HI), and the division as a whole is also moving towards a more equal gender balance. 

\section{Satellite data}
%
All satellite data of concern will be publicly available, as coming from
operational weather sensors.

{\footnotesize
	\bibliography{j_abbr,refs_pe1,references}
}
\end{document}
