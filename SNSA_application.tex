\documentclass[12pt,oneside,a4paper]{article}
%
\usepackage{graphicx}
\usepackage{tabularx}
\usepackage{url}
\usepackage[latin1]{inputenc}
%\usepackage[utf8]{inputenc}
\usepackage[left]{eurosym} 
\usepackage{times}
\usepackage{pdfpages}
\usepackage{todonotes}
\usepackage{enumerate}

% We don't want a special font for urls (looks bad with times):
\urlstyle{same}

% Graphics extensions and path:
%\DeclareGraphicsExtensions{.pdf,.png,.jpg}
%\DeclareGraphicsExtensions{.eps,.ps}
%\graphicspath{{figures/}}

%
% Page size
%
\usepackage[top=25mm,left=25mm,right=25mm,bottom=25mm]{geometry}

\setlength{\headheight}{15pt}    % Necessary to avoid fancyhdr warning.


\newcommand\LongTitle{Humidity Retrievals over Arctic}
\newcommand\ShortTitle{Arctic Humidity}



%
% Heading
%
\newcommand{\pagevers}[2]{
\ifnum\thepage=1 
#1
\else#2
\fi
}
%
\usepackage{fancyhdr}
\pagestyle{fancy}
\chead{}
%\rhead{\pagevers{}{\bf \thepage}}
\rhead{\thepage}
\rfoot{\small \it \ShortTitle\ ---\  Application to SNSB 2021-R}
\cfoot{}
\lfoot{}
%\renewcommand{\headrulewidth}{\pagevers{0pt}{0.2pt}}
\renewcommand{\headrulewidth}{0.2pt}
\renewcommand{\footrulewidth}{0pt}


\newcommand{\docname}[1]{\lhead{\small #1}}

%
% Section titles
%
\usepackage[small]{titlesec}
\titlespacing*{\section}{0pt}{*3.3}{*0.5}
%
% 
%
\def\compactitems{\parskip0pt\topsep0pt\partopsep0pt\parsep0pt\itemsep0pt}

% Struts for better table formatting:
\newcommand\T{\rule{0pt}{2.6ex}}
\newcommand\B{\rule[-1.2ex]{0pt}{0pt}}


\newcommand{\FIXME}[1]{{\bfseries \textcolor{red}{FIXME:} #1}}
\newcommand{\md}[1]{\mbox{#1-d}}
%\newcommand{\3d}{3d}

%\hyphenation{3-d}
\uchyph=0

%%% Local Variables: 
%%% mode: latex
%%% TeX-master: t
%%% End: 



\docname{Project description}
\usepackage{graphicx}
\usepackage{caption}
\usepackage{subcaption}
\usepackage{multirow}
\usepackage{amsmath}
\usepackage{booktabs}
\captionsetup[figure]{font=small,labelfont=small}

%
% References
%
\usepackage{natbib}
\bibliographystyle{agu04}     
\setlength{\bibsep}{0mm}


\newcommand\wpstart[3]{\noindent\textbf{WP #1, #2}\hspace{\stretch{1}}Priority #3%
	\vspace{-4mm}\\\rule{\textwidth}{0.5pt}\\}
\newcommand\wpenda[4]{%
	\noindent -----\\ 
	\begin{tabularx}{0.95\hsize}{l p{133mm}}    
		\hspace*{-1.1ex}Start\,--\,end: & #1\\
		\hspace*{-1.1ex}Main output: & #2\\
		\hspace*{-1.1ex}Main risks: & #3\\
	\end{tabularx}\\
	\vspace{-2.2ex}\noindent\rule{\textwidth}{0.5pt}\\
}


\newcommand\intodo[1]{{\color{red} [* #1 *]}}


\begin{document}
	
	
	\thispagestyle{empty}
	\vspace*{-10mm}
	\noindent
	\textbf{\Large \LongTitle}




\section{General summary}
%
The Arctic is a water world, experiencing rapid warming. Passive microwave
satellite measurements are required for a comprehensive view of the water in
the region and the basic aim of the project is to improve the utilisation of
this data source, particularly the high-frequency part. The project targets
water in the atmosphere. The primary concern is humidity, but the retrieval
system is highly versatile and retrievals of liquid water content,
snowfall, and sea-ice concentration (in order of priority) are also feasible.

We propose a physically-based, Bayesian-oriented, machine-learning model for
the task. The model is trained with detailed 3D radiative transfer simulations,
to generate local scenes of measured radiances. The input to the simulations
has a horizontal resolution finer than the size of the satellite footprints.
For example, high-resolution SAR (Synthetic-aperture radar) imagery is applied to describe the
distribution of sea-ice\,/\,open water at a resolution of \intodo{$\sim$90\,m}. As a result,
aspects like inhomogeneous filling and overlap of the footprints are
incorporated into the simulations and are then automatically considered in the
retrieval process, in contrast to existing data extraction. Precipitation and
clouds are modelled in detail by combining a mesoscale model with the in-house
expertise on microwave radiative transfer involving hydrometeors. As we cover
all significant parts of the ``forward problem'', our inversions can be truly
``all-sky''; we aim for a zero data rejection (for the set of channels
included).

The resulting dataset(s) would play an important role in not only analysing the
variability of water above the Arctic, but also in deciphering the
ongoing trends involved in the Arctic amplification of global warming.


%Water vapour plays an important role in the water and energy cycle over the polar regions, especially over Arctic. Over the Arctic, changes in the water vapour content are most sought to observe the warming trends, also know and Artic amplification. However over poles, the ground measurements are sparse and the performance of the satellite retrievals is constrained by the highly variable surface emissivity. The basic aim of this project is to develop new retrieval algorithms principally for total water vapour (TWV) from satellite microwave radiometer data over the Arctic. We propose a physically based retrievals, using a Bayesian machine learning based inversion method. Additionally, this algorithm could be used to also retrieve side products like total liquid water content (LWC) and sea-ice concentration (SIC). A key feature of this algorithm would be the use high resolution SAR imagery to classify sea-ice and open waters. Fine scale sea-ice information is necessary for distinguishing emissivity between different surface types. Another crucial aspect of the development line will be to use data from multiple microwave frequencies, with different footprint size, and avoiding the remapping of data. The overall scheme shall be similar to the one in a parallel SNSA 2021 proposal submitted by co-applicant Patrick Eriksson.
%The resulting dataset would play an important role in not only analysing the water vapour variability of the Arctic atmospheric but also in deciphering the trends the Arctic climate change.


\subsection{Background}
%
\label{sec:background}
The Arctic is a very sensitive region to climate change and has a significant
influence on the mid-latitudes, e.g.\ European weather patterns. In the recent
years an amplified warming (2-3 times higher compared to the rest of the world)
has been observed in this region (Arctic amplification), which is caused by
unique feedback processes influencing the Arctic climate. Unfortunately
detailed knowledge on the feedback processes is still missing and the scarcity
of observations in the Arctic makes it difficult to assess the situation and
processes taking place. However, water vapour and the increase in water vapour
in the last decades due to decreased sea ice cover seems to be one of the key
players of Arctic feedback processes \citep{serreze:2012:recen,
  vihma:2016:theat} A decrease in sea ice leads to an increased uptake of solar
radiation of the Arctic ocean which enhances the heat flux from the ocean to
the atmosphere and the amount of water vapour in the air, a greenhouse gas that
can induce further warming and melting of sea ice \citep{screen:2010:thece}. In
addition to its role as a greenhouse gas, water vapour is also relevant when it
comes to the water cycle in the Arctic, i.e. clouds and precipitation which
have an impact on surface temperature and sea ice as well
\citep{blanchet:water:1995}. To understand and monitor water vapour in the
Arctic, its role for the Arctic water cycle and its climate impact,  observations
are needed. Conventional observations such as radiosondes have a substantial
impact on the numerical weather prediction of the Arctic
\citep{lawrence:2019:usean}, but they are not able to provide a comprehensive
spatial coverage as satellites can do.

\subsection{Previous work}
%
\label{sec:previousworks}
%
Over the Arctic, polar orbiting satellites provide the most dense coverage
of observations, both by infrared and microwave instruments. As the area
experiences polar night and has extensive cloud cover, optical and infrared
techniques suffer from basic limitations; thus to avoid these issues, only
satellite-based passive microwave data are considered. Over the polar regions, a
challenge encountered by passive microwave observations is the high and highly
variable contribution by surface emission.

The most extensive work towards stand-alone retrievals of water vapour path
(WVP) from microwave humidity sounders (such as Advanced Microwave Sounding
Unit-B (AMSU-B) and Microwave Humidity Sounder (MHS) comes from University of
Bremen. Their retrieval concept was initiated by \citet{miao:2001:atmos}, where
they utilized water vapour absorption channels around 183\,GHz and 150\,GHz
window channel to retrieve the WVP upto 7\,kg m$^{-2}$. Subsequently
\citet{melsheimer:2008:impro} extended this approach to include 89\,GHz to
retrieve WVP upto 15\,kg m$^{-2}$ over sea-ice regions. They used data from
measurement campaigns to formulate a relationship between sea-ice emissivity
over different frequencies. Later, \citet{scarlat:2018:retri} extended the
method to include all surface types using AMSU-B. A comparison of the retrieved
WVP against ERA-Interim showed that the over winter months, the RMSD was
1.86\,kg m$^{-2}$ but over summer months the errors were up to 5.67\,kg
m$^{-2}$ due to the algorithm being constrained by its upper retrieval limit of
15\,kg m$^{-2}$. Further, an attempt has also been made to use low
frequency microwave observations from Advanced Microwave Scanning Radiometer
(AMSR). For example, \citet{scarlat:2017:exper} use optimal estimation (OEM)
for multi-parameter retrieval over Arctic, and \citet{zabolotskikh:2020:anadv}
attempt at WVP retrieval over both open ocean and sea-ice regions using neural
networks based inversion. In both products, the highest uncertainties in the
retrieval are linked to the empirical estimates of surface emissivity over
sea-ice regions. In fact the skill of all available satellite based water
vapour retrievals in this region is highly variable over different atmospheric
and surface conditions. For example, in the central Arctic, for summer months,
the monthly means derived from different satellite products can differ up to
30\% \citep{crewell:2021:asyst}.

The frequent sampling of polar orbiting satellite data over the Arctic caries
huge potential at improving numerical weather prediction (NWP).
\citet{lawrence:2019:usean} show that over summer months, assimilation of
microwave sounding observations leads to more than 3\% reduction in the total
global forecast error, but during winters, their impact reduces to 2\%, as
almost 24\% of the microwave observations are rejected. The high rejection rate
in the winter season is a consequence of the forward model errors associated
with the estimation of surface properties particularly over snow and sea-ice,
and the high forecast model errors \citep{bauer:2016:aspec}. Thus, efforts are
required to bridge the gap between the seasons. Further, despite that
reanalysis based on NWP show good skill in for WVP, vertical profiles can be
systematically wrong and cloud fractions are poorly predicted, with large
impact on the surface radiative fluxes driving melting of ice
\citep{graham:2019:evalu}. Increasing the usage of satellite data over all
surfaces in Integrated Forecast System (IFS) is also identified as one of the
priorities in ECMWF (European Centre for Medium-Range Weather Forecasts)
Strategy for 2021-2030.



\subsection{The way forward: Problems and solutions }
%\subsubsection{Problems and solutions}

Remote sensing provides indirect measurements and the data retrieval always offers some degree of complexity. When using passive satellite microwave data to estimate humidities over the Arctic region, as mentioned, the main limiting factor is the surface's contribution to measured radiances. The special role of the surface is due to high atmospheric transmissivities and the difficulties to predict the surface properties of snow (on either land or ice).
The main solution today is to reject channels with a significant surface
contribution. This gives a low utilisation rate of the satellite data, and
leaves the humidity close to the surface unconstrained.

For this reason, efforts are being made to improve the "forward modelling",
i.e.\ use auxiliary data to predict the local radiative properties of snow and
sea-ice. This constitutes a very hard problem
as the properties depend on a high number of snow and sea-ice
variables. There will be progress in the area, but it will likely be slow.
There is also another, less discussed, consideration, that the satellite
footprints can cover both sea ice and open water. To handle this, a large scale
ice-fraction is not sufficient, even the distribution of ice and water inside
the footprint matters. This makes retrievals above leads particularly
problematic. The risk of having an inhomogeneous footprint increases with
footprint size, and good spatial resolution is thus advantageous even if the atmospheric fields show little horizontal variability.



We will attack these issues from a new angle, now feasible due to progress in
machine learning (ML). We avoid the limitations in traditional approaches by
basing the inversions on a retrieval database, that is used to train the ML
model. The simulated measurements in the database are generated on
high-resolution data and variations of both surface and atmospheric variables
inside the footprint get included. In addition, by simulating "scenes" of
adjacent footprints, and not just individual ones, the information hidden in
the overlap of footprints can be exploited. The later effectively increases the
horizontal resolution, but also provides background spatial information, such
as indications on if the surface is homogeneous or mixed.


\section{Project description}
%
\subsection{Scientific objectives and considerations}
The main objective of this study is to provide physical, stand-alone, water
retrievals over the Arctic from microwave instruments, which can be comparable
to or better in performance than existing reanalyses and satellite based
products. The basic idea is to combine a database based on sophisticated 3D
radiative transfer with ML to retrieve water properties and estimate the
associated uncertainties. The primary aim is to estimate humidity, however, a
simultaneous retrieval of liquid water content will also be undertaken as a
second priority. Nonetheless, before going ahead, it is crucial to elucidate
the following:
\begin{itemize}
 \vspace{-1ex}
\item The region north of Arctic Circle ($66^{\circ}33^{'}$N) will be defined
  as the Arctic region.
 \vspace{-1ex}
\item The retrieval database will work like the prior assumptions in a standard
  Bayesian retrieval, but by using ML we will not be restricted to Gaussian
  statistics. ML can handle strong non-linear problems, and we can include
  aspects that are very difficult to handle in traditional retrievals (such as
  inhomogeneous footprints). It is not needed to predict the conditions exactly
  where the observation is made, it is sufficient for the simulations to have a
  similar variability as reality. The ML algorithm uses the database provided
  to give the posterior knowledge of the quantity sought, given the scene of
  observations.
 \vspace{-1ex}
\item We will invert all observations. For certain conditions some channels
  will not provide any useful information due to interference of the surface or
  e.g.\ heavy snowfall. Our retrievals will then fall back to the prior
  information embedded in the database in a stable manner and a proper
  uncertainty estimate can still be given. The key here is that the disturbing
  factors are included in the simulations. This in contrast to other retrievals
  where forward model errors not are properly described and e.g.\ including
  surface sensitive channels can easily lead to a totally unphysical retrieval.
  \vspace{-1ex}
\item If useful additional information becomes available during the course of
  the project, it can be incorporated into the ML training. For example, if
  there is progress in snow modelling and the local emissivity can be estimated
  with some certainty, this can be seen as a virtual measurement and be used to
  improve the ML model.
\end{itemize}


\subsection{Data and tools}
% 
\subsubsection{Satellite instruments}

As introduction it is clarified that the retrievals developed here will not
only be applicable to existing sensors, but also to upcoming satellite
instruments like MicroWave Imager (MWI) and Arctic Weather Satellite (AWS). MWI
will be part of the next generation of the European Organisation for the
Exploitation of Meteorological Satellites Polar System - Second
Generation\footnote{See further
  \url{www.wmo-sat.info/oscar/instruments/view/mwi_metop_sg}}. AWS is an ESA
mission and is envisaged as a constellation of small polar-orbiting satellites,
providing measurements at a high temporal resolution\footnote{See
  \url{www.esa.int/Applications/Observing_the_Earth/Meteorological_missions/Arctic_Weather_Satellite}}.
AWS has its origin in an initiative by SNSA, and Sweden is also the largest
contributor to the funding of the initial satellite. A brief description of the
relevant instruments is provided in Table~\ref{tab:specifications_instruments}.

As a first step, the methodology will be applied to SSMIS (Special Sensor
Microwave
Imager/Sounder\footnote{\url{www.wmo-sat.info/oscar/instruments/view/ssmis}}).
This is presently the only conically scanning microwave radiometer covering the
Arctic having channels around 183\,GHz (it will be followed by MWI). There is a
higher selection of possible cross-track sensors. Here ATMS (Advanced
Technology Microwave
Sounder\footnote{\url{www.wmo-sat.info/oscar/instruments/view/atms}}) will be
the primary choice as it has 5 channels around 183\,GHz (as both MWI and AWS
will have). The retrieval scheme is basically identical between conical
scanners and cross-track scanners, but the latter will require more simulations
(for the retrieval database) to cover the varying incidence angle for this
scanning approach.

\begin{table}[!t]
	\footnotesize
	\centering
	\caption{Brief specifications of the conical scanners (left) and cross-track scanners (right), relevant to this study.}
	\label{tab:specifications_instruments}	
	\parbox{.45\linewidth}{
	\centering
	\begin{tabular}{crr}
	\toprule
		Instrument & Frequency range 	& Footprint size \\
					& [GHz]             & [km]       \\
		\midrule			
		SSMIS	   &19 - 22		& 42.4$\times$70.1	\\
				   &37          &27.5$\times$44.2  \\
				   &50 - 63       & 17.5$\times$25.8 \\
				   &91 - 183    &  13.1$\times$14.4\\
		\midrule
		MWI 	   &18-23 		&50\\
				   &31-53 		& 30\\
				   & 89-183 	& 10\\	
		\bottomrule		
	\end{tabular}
	}
\hfill
\parbox{.45\linewidth}{
	\centering
	\begin{tabular}{crr}
	\toprule
	Instrument & Frequency range 	& Footprint size \\
	& [GHz]             & [km]       \\
	\midrule			
	ATMS	    &23 - 32		& 75	\\
				&50 - 90        &32  \\
				&165-183        & 16 \\
	\midrule
	AWS 	   &50 - 57 		& $\le$40\\
			   &89 				& $\le$20\\
			   & 165 - 325 		& $\le$10\\	
	\bottomrule		
\end{tabular}
}
\end{table}



\subsubsection{HARMONIE-AROME}
%
\label{sec:harmonie}
The first European regional reanalysis focusing on the Arctic regions is made
by the HARMONIE-AROME numerical weather prediction (NWP) system. It is one of
the canonical model configurations of the ALADIN-HIRLAM NWP system
(\url{http://www.umr-cnrm.fr/aladin/}) and is also used for operational
short-range NWP over Northern Europe. The reanalyses provides data coverage
over two domains of the European Arctic, and includes the four largest bodies
of the Arctic land ice. The reanalyses cover the period from 1997 to 2021 and
are available at 2.5\,km horizontal resolution, thus providing more local
detailing than ERA5 global reanalysis products. Details about the model
microphysics, parametrizations and other configuration choices can be found in
\citet{bengtsson:2017:harmo}.
 
\subsubsection{QRNN}
%
\label{sec:qrnn}

Quantile Regression Neural Network (QRNN, \citet{pfreundschuh:aneur:18}) is the
ML approach to be applied and it operates in a Bayesian manner. It can be seen
as a ML version of Bayesian Monte Carlo integration (BMCI) to solve ill-posed
problems. QRNN predicts the posterior distribution over the chosen quantiles
and provides robust case-specific error estimates. That is, uncertainty is
assigned to each individual retrieval.

%The neural network (NN) training is a process of learning to predict the outputs {$y_i$} from inputs {$x_i$} through a series of learnable transformations. In traditional NN techniques, the output is a point estimate of the target variable. However, QRNN is trained to minimise the mean of the quantile loss function and predict chosen quantiles of its Bayesian a posterior distribution.   

In all the applications QRNN has been tested so far, it has outperformed the
existing approaches. This includes a very recent study by Inderpreet Kaur for
predicting noise-free clear-sky radiances from microwave humidity channels
\citep{kaur:2021:canma}. Previously, \citet{pfreundschuh:aneur:18} had shown
the advantage of QRNN in predicting cloud top pressure from measurements by the
Moderate Resolution Imaging Spectroradiometer (MODIS). Ongoing studies with
QRNN include working with Goddard profiling algorithm \citep[GPROF,][]
{kummerow:2015:thevo} team to replace BMCI with QRNN in the GPROF retrievals
(manuscript in preparation)


\subsubsection{ARTS}
\label{sec:arts}
% 
The backbone of the work is to genarate the radiances of the database. These
are simulated by ARTS \citep[Atmospheric Radiative Transfer Simulator,][]
{eriksson:arts2:11}. ARTS has some unique features, but in this context,
it is rather the completeness and flexibility of ARTS that is helpful. For
example, it includes several tools for calculating the surface emissivity
(FASTEM, TESSEM and TELSEM, SMRT will be added) and can retrieve the same
emissivity by 1DVAR retrievals. To simulate full radiance scenes is
straighforward as ARTS can operate in 3D with full description of antenna
patterns \citep{duncan:anexp:19}.

There is now a second cornerstone of the ARTS infrastructure, the associated
database of single scattering properties \citep{eriksson:agene:18}. The main
part contains data for 36 particle ``habits'' assuming totally random
orientation (TRO). This makes the database the most comprehensive one. Some
data for azimuthally random orientation (ARO) are also at hand
\citep{brath:micro:20}. In principle, more ARO data of ice hydrometeors are
needed, but in \citet{barlakas:intro:21} we show that the ARO case can be
fairly well approximated by scaling the TRO data.

\subsection{Preliminary results}
%
This project will build upon the ongoing efforts, hence we briefly summarize
the preparatory steps we have undertaken to demonstrate the feasibility of the
project. The initial steps are based on the measurements from the Global
Precipitation Measurement (GPM) Microwave Imager (GMI). GMI is
non-sunsynchronous conical scanning microwave radiometer providing coverage
upto 65$^{\circ}$. We select data from January and only high latitudes
(45$^{\circ}$ - 65$^{\circ}$ in both hemispheres), so that a varietry of
sea-ice, snow, ocean and land surface types are included.

\subsubsection{Radiative transfer simulations}
%
\label{sec:radiative_transfer}
The simulations cover the four 166 and 183\,GHz channels of GMI. The dbZ based
system described in \citet{ekelund:using:20} is followed. The
main inputs are radar reflectivities from Cloudsat and ERA5 reanalysis data.
Hydrometeor particles are assumed to be ARO (Sec.~\ref{sec:arts}), where the
approximation from \citet{barlakas:intro:21} is extended to operate with 
a random scaling factor (instead of a fixed one), as well as being applied also
on the Cloudsat radar data.

The emissivities over land and water are taken from climatologies. However, for
snow and sea-ice surface types (identified using ERA5 data) an empirical snow
and ice emissivity model was developed, based on the studies of
\citet{harlow:2009:milli} and \citet{hewison:2002:airbo}. If
$\epsilon_{193}, \epsilon_{159}$ represents the emissivities for 193\,GHz and
159\,GHz respectively, then
\begin{align}
\epsilon_{193}& = \min({N(\mu_{193}, \sigma_{193}^{2}), 1});\, \mu_{193} = 0.78, \sigma_{193} = 0.07 \label{eq:1}\\
\epsilon_{159}& = \min(\epsilon_{193} - N(\mu_{159}, \sigma_{159}^{2}), 1) ;\,  \mu_{159} = 0.02, \sigma_{159} = 0.02\,\label{eq:2}
\end{align}
where, $N(\mu, \sigma^{2})$ represents the standard normal distribution with
mean $\mu$ and standard deviation $\sigma$. The differences between the
horizontal and vertical polarisations for both frequencies are 
approximated through a uniform random distribution.

%\begin{align}
%d_{159}& = U(a_1, b_1) ;\, a_1 = 0.005, b_1 = 0.055\\
%d_{193}& = d_{159} - U(a_2, b_2) ;\, a_2 = 0.015, b_2 = 0.025 \,
%\end{align}
%where, $U(a, b)$ represents a uniform distribution between a and b. 


\subsubsection{Training database}
%
\begin{figure*}[t]
	\centering
	\begin{subfigure}{.24\textwidth}
		\caption{Land}
		\includegraphics[height=39mm, width = 39mm]{Figures/hist2d_gmi_45-65_land.pdf}
	\end{subfigure}
	\begin{subfigure}{.24\textwidth}
		\caption{Water}
		\includegraphics[height = 39mm, width = 39mm]{Figures/hist2d_gmi_45-60_sea.pdf}
	\end{subfigure}
	\begin{subfigure}{.24\textwidth}
	\caption{Snow}
	\includegraphics[height = 39mm, width = 39mm]{Figures/hist2d_gmi_45-60_snow.pdf}
\end{subfigure}
\begin{subfigure}{.24\textwidth}
	\caption{ Sea-ice}
	\includegraphics[height = 39mm, width = 39mm]{Figures/hist2d_gmi_highlat_sea-ice.pdf}
\end{subfigure}
\caption{Two-dimensional occurence frequencies of 166\,GHz GMI data. The x-axis
  is the brightness temperature of the channel having vertical polarisation.
  The y-axis is the difference to the channel having horisontal polarisation.
  Figures (a)-(d) are for land, water, snow and sea-ice surface types,
  respectively. The data are from the 45$^\circ$ - 65$^\circ$ latitude bands in
  both hemispheres. In panel c, the parts of the distribution most closely
  connected to impact of the surface and hydrometeors are marked.\intodo{Remove
    ``clear''}}
  \label{fig:histogram_2d}
\end{figure*}

The training database must represent the actual measurement space, otherwise,
the retrievals associated with poorly represented cases would be inaccurate.
Accordingly, before starting with the retrievals, it is crucial to verify the
distribution of simulated measurements in the training database.
Figure~\ref{fig:histogram_2d} shows radiance histograms for the two 166\,GHz
channels, for both simulated and measured data. For all the surface types
shown, the histogram contours for both datasets overlap to a large extent. This
indicates that we simulate the polarisation signatures of all surface types and
ice hydrymeteors well. There is a similar good match with observations for the
183\,Ghz channels (not shown). This is an important result as we cannot expect
the retrieval to work properly unless the training database mimics the real
observation quite closely.


\subsubsection{Retrievals}
\label{sec:preliminary_results}
\begin{figure*}[t]
	\centering
	\begin{subfigure}{.54\textwidth}
		\includegraphics[height = 55mm]{Figures/WVP_spatial_jan2020.png}
	\end{subfigure}
	\begin{subfigure}{.34\textwidth}
	\includegraphics[height = 50mm]{Figures/WVP_scatter_monthlymean.png} 
	\end{subfigure}
	\caption{The spatial distributions (left) of monthly means estimated from retrieved WVP and ERA5 WVP. The data is for January 2020. The corresponding scatter between the two datasets is shown on the right. The red line is the line of best fit, while the black indicates the perfect fit.}
	\label{fig:WVP_retrievals}
\end{figure*}

The QRNN algorithm (Sec.~\ref{sec:qrnn}) is trained with radiances from
simulated GMI data for the frequencies: 166 V\,GHz, 166 H\,GHz, 183$\pm$3\,GHz
and 183$\pm$7\,GHz, to retrieve the corresponding WVPs. Two-metre temperature,
latitude and surface type are also provided as auxiliary information. The
trained net is then used to retrieve the WVP using actual GMI measurements. A
comparison of monthly means of the retrieved WVP (re-gridded at 2.5$^{\circ}$
resolution) and ERA5 reanalyses is shown in Fig.~\ref{fig:WVP_retrievals}. The
spatial distributions of both datasets largely agree with each other, except
around 45$^{\circ}$S. The presence of these erroneous cases is not surprising,
they likely occur due to the saturation of high frequency channels in regions
with high humidity. This saturation effect will be lower when including further
channels. In spite of the outliers, the correlation between the two datasets is
0.97.


\subsubsection{Sea-ice classification using SAR}
\todo{from Leif}



\subsection{Work Flow}
\label{sec:wp}


The work flow is described as work packages (WPs) and the overview is displayed
in Fig.~\ref{fig:flowchart}. No Gantt chart is included as it is would not be
informative. The WPs will overlap and extend over large fraction of the project
period. The approach will be to have a first retrieval version ready early (at
least during year 1), analyse the results and then repeat parts of the first
WPs were found needed. A likely situation is that the retrievals will work less
well for one surface type, or that including some channels deteriorate the
accuracy, and there will be reasons to reiterate one or several WPs.

The project will require extensive calculations, but
the computational resources we have available by the Chalmers central cluster
(C3SE) and locally inside the division still meet the demands with margin.

\begin{figure}
	\begin{minipage}[c]{0.70\textwidth}
		\includegraphics[trim=140 450 25 125,clip,height = 50mm]{flowchart.pdf}
	\end{minipage}\hfill
	\begin{minipage}[c]{0.28\textwidth}
		\caption{Flowchart of the proposed retrieval system. The green boxes
          represent the retrievals. 
		} \label{fig:flowchart}
	\end{minipage}
\end{figure}

\vspace{-1ex}
\subsubsection*{WP 1: Basic atmospheric scenarios}
%

\label{sec:atmscenes}
The inputs to the forward model can be based on either atmospheric models
providing a sufficiently detailed description of hydrometeors, or on
measurements \citep{ekelund:using:20}. In this study, we will adopt the former
approach and use reanalyses from HARMONIE-AROME to create a database of the
atmospheric cases. HARMONIE-AROME does not cover all the Arctic regions, so if
needed, reanalyses from ERA5 could be included. Furthermore, use of
high-resolution SAR imagery to characterize the sea-ice/open waters will be
made. The fine spatial resolution of SAR will facilitate better mapping of
inhomogeneous surface types, and thus reduce errors with respect to mixed
surface types such as polynyas. To our best knowledge, this shall be the first
instance where such high-resolution information would be used in this context.
The output of this work package will be a broad set of 3D atmospheric scenarios
to be used as input to the radiative transfer simulations. \vspace{-1.0ex}

\subsubsection*{WP 2 : Emissivity model}
%
\label{sec:emissivity}
An important step to accomplish this task is the estimation of the surface
emissivity spectra over a multitude of surface types. As a first step, we will
consider using existing studies to extend the snow-emissivity model (see
Eqs.~\ref{eq:1}- \ref{eq:2}) to lower frequencies such as 89\,GHz and 50\,GHz.
Such an empirical model will be easy to fine-tune by comparing observed and
forward modelled radiances. Also OEM-based (with ARTS) retrievals will be
explored. A joint retrieval of the emissivity spectrum over all frequencies
will be made. The basic idea is to find a set of emissivity estimates for all
the considered frequencies that would simultaneously give radiance closest to
the measurements. These retrievals should be especially useful to understand
the correlation of emissivity variations between frequencies; thus making it
more viable from an assimilation perspective. The retrievals in this WP will be
made assuming both specular and lambertian reflection. The differences between
the two more or less vanish for conical scanners, while for cross-track
scanners a combination of the two have found to be a realsitic assumption
\citep{matzler:2005:onthe}. Tests will be made to judge exactly what
combination works best for various surface types and incidence angles.

The output of this work package will be an emperical, statistical, model of
snow emissivities. If found needed there will be two models, one for snow on
land and one for snow/sea-ice. This WP could potentially result in a separate
journal article.
\vspace{-1.0ex}

\subsubsection*{WP 3 : Database creation and post-processing}
%
\label{sec:database}	
This WP covers the initiation and execution of the batch radiative
transfer calculations. Initially, some time will be spent on to implement the
scripts needed for the basic calculations, to add jobs to the calculation
cluster, and post-processing. As a start, only conical scanners will be
simulated, and later extension to cross-track scanners will be made. For the
latter, a number of scan angles, between nadir and swath edge, will be covered.

ARTS will be applied to generate simulated observations, based on the data and
models from WP 1 and 2. For both types of scanners, overlapping antenna
footprints of all frequencies will be explicitly simulated. To generate the
antenna pattern at the desired frequency, a number of pencil beam radiative
transfer calculations will be performed to incorporate both along-track and
across-track sampling. To also accound for imperfect calibration of the
instruments, bias adjustments as calculated by ECMWF could be considered as a
post processing step. Sensor calibration and its relative agreement with the
forward model are vital for the success of the retrieval database. The WP
output will be batches of simulated radiances. \vspace{-1.0ex}


\subsubsection*{WP 4 : Retrieval setup}
%
\label{sec:setup}
The retrievals will be made without introducing any re-mapping of the satellite
measurements. All footprints inside a region would be used as input, following
the size of the simulated scenes in the database generated in WP3. This
approach will make much better use of the spatial information provided by the
overlap of instrument footprints. The target resolution would not be
compromised and retrievals can be made at the highest possible resolution,
while preserving the information at their native resolution. This has been
demonstrated using 2DVAR in \citet{duncan:anexp:19}. But here we shall aim to
achieve the same purpose by using QRNN. In fact, by doing this by a database
(and not 2DVAR) we are not restricted to describe spatial correlations by
covariance matrices and the QRNN version should be even more advantageous.

QRNN is already available as a part of the Typhon software package
\citep{lemke:2020:typhon}. The implementation of the retrieval algorithm shall
be relatively straightforward but it will be important to find the
neural-network architecture required to achieve the best model performance.
Additionally, the retrieval sensitivity to different auxiliary data would be
analysed.


\vspace{-1.0ex}
\subsubsection*{WP 5 : Retrievals}
%
\label{sec:retrievals}
%
In this WP we will extend our preliminary results for WVP (see
Sect.~\ref{sec:preliminary_results}) to include regions north of 65$^\circ$,
using SSMIS. An extension to lower frequencies, such as 89\,GHz and temperature
sounding channels around 50\,GHz, will be tested. Including the latter can also
be beneficial to resolve the low-level cloudiness and hence improve the
humidity estimates. In fact, the estimates of LWC and WVP are not completely
independent. The synergies between the two quantities can be exploited to
correct the estimates for each other. For LWC over sea-ice, the combination
89+150\,GHz have been suggested by \citet{laue:2007:poten} to cancel out the
effect of emissivities. Inclusion of 22\,GHz and 37\,GHz to increase
sensitivity to high humidity regions could also be considered, though on a
lower priority. These channels are more important over open-waters than sea-ice
regions. Later, the best channel combinations will also be used to estimate the
humidity profiles. The ultimate goal is to include sufficient channels such
that the full information content of tropospheric water is preserved. Extension
to cross-track scanners will also be made. The same scheme will be applied to
cross-track scanners however, the scheme will be modified to include the
changes in surface emissivity with incidence angles. A larger database covering
the various incidence angles will be necessary, but we will try to focus only
on the central part of the swath, and avoid working with the increasing field
of view towards the edge of the swath.
%The pivotal part of these retrievals would be using ML on the overlapping footprints to extract the hidden information without any explicit treatment.

We will start the test retrievals in parallel to database creation so that
early feedback on any issues can be made. QRNN is versatile to allow extension
to include additional input channels or auxiliary data and the main task will
be to analyse the results. During the development phase, evaluation (see
Sect.~\ref{sec:evaluation}) of the retrievals with existing retrieval products
will be made to identify the areas requiring improvement.

The output of this WP will be one or more retrieval databases for WVP, humidity profiles and LWC.
\vspace{-1.0ex}
\subsubsection*{WP 6: Evaluation and dissemination}
%
\label{sec:evaluation}
Since in this project we aim to solve the retrieval issues associated with the
ever complex and changing surface characteristics of sea-ice, a detailed
assessment will be expected to improve the applicability and future retrieval
algorithms. A thorough investigation with the available \textit{in-situ} data,
satellite products and reanalysis will be made. Firstly, direct comparisons
with radio-soundings or data from flight campaigns will be used to estimate the
accuracy of the product and reveal the uncertainties on finer scales.
Statistical comparisons with other microwave and IR based humidity products
will also be made. Time series of daily and monthly means, and the probability
distribution functions of the retrievals will be analysed. Here it would also
be important to analyse the accuracy over seasons and different surface
types.The evaluation part shall also go hand-in-hand with the retrieval WP.

The dissemination of the retrieved dataset will be through conferences and journal articles. The assessment of how well the produced dataset answers the needs of NWP communities will also be addressed through collaborations (see Sect.~\ref{sec:collaborations}) with ECMWF and Swedish Meteorological and Hydrological Institute (SMHI). 
\vspace{-1.0ex}
\subsubsection*{WP 7 : Additional Retrievals}
%
\label{sec:other_retrievals}

Secondary products which will be considered include snowfall and sea-ice concentrations.  

\todo{snow}

The high-resolution emissivity retrievals can be additionally post-processed to generate estimates of sea-ice concentration. The post-processing simply involves matching the 1D-VAR emissivity estimates over sea-ice to a pre-computed catalog of mean emissivity spectra. %In fact, validation of the SIC against other available products like MiRS (Microwave Integrated Retrieval System) based operational cyrogenic products can also serve as an indirect validation of the emissivity retrievals.

 
\subsection{Risk Assessment}
%
\label{sec:risk}
All activities planned in this project are a direct continuation of ongoing
work and the level of risk in the basic work plan must be judged as low. The
calculation of emissivity spectra over sea-ice and heterogeneous regions is
often considered difficult and neglected, however with use high resolution SAR
data (as planned) the risk of contamination and unwanted errors becomes low.
Usage of low frequencies like, 22 and 37\,GHz should be seen as more
speculative considering their large footprint size, but their inclusion has
also low on priority. As mentioned, existing access to computational resources
meet the demands raised by the project.


\section{Impact and social benefits}
%
\label{sec:impact}

As a society, understanding and improving the weather, environmental and
climate processes in the polar regions are important to tackle the effects
of climate change, and also improving mid-latitude weather forecasts. The
anthropogenic climate change over poles also has economic and political
implications, e.g.\ commercial navigational activity through the Arctic
requires accurate weather forecasts to avoid accidents and oil spills.

The overall retrieval procedure requires a deep understanding of the physics
governing the observed radiances. This knowledge can be shared with the NWP
community and an indirect significant impact of the data on weather forecasting
can be expected. Work towards amplifying the ingestion observations in NWP will
be covered in association with modelling community at SMHI and ECMWF (see
Sect.~\ref{sec:collaborations}).


%A consistent database of standalone retrievals
%is important for NWP, which rely on satellite based retrievals to determine the
%initial atmospheric conditions. Additionally, the sea-ice emissivity estimates
%can be used to reduce the forward model errors arising due to surface
%contribution in radiative transfer.


\section{Collaborations}
%
\label{sec:collaborations}
The retrievals will be made publicly available and we will actively search
for cooperation inside the climate modelling community to ensure that the 
data will be applied for verification of models and similar activities. \intodo{* Ask
Luisa for names to mention?}

The experience obtained in the project can be transferred to the NWP community
by existing contacts. Patrick Eriksson has an established collaboration with
SMHI (around ICI and QRNN), and AWS should broaden the contact. For example,
discussions are ongoing on how Chalmers can contribute to the upcoming
``Nordic AWS study''. Further, Patrick and Leif Eriksson are co-applicants in a
proposal from SMHI, also submitted to SNSA 2021-R. The title of that proposal
is ``Consistent Air-Ice-Sea Data Assimilation of Satellite Observations`` and
there are high similarities between the projects.

The EUMETSAT fellowship hold by V.\ Barlakas has established a direct
collaboration with ECMWF \citep{barlakas:intro:21}. Furthermore, the interest
in the research performed at Chalmers from ECMWF has resulted in
that Patrick
Eriksson has been appointed ECMWF fellow\footnote{See 
  \url{www.ecmwf.int/en/about/media-centre/news/2020/ecmwf-appoints-seven-new-fellows-scientific-collaboration}}.
Another contact at ECMWF is through David Duncan (former post doc at Chalmers).
He contributed to the recent study by Inderpreet Kaur, and this contact will be
maintained.
\section{Staff situation}
%
\label{sec:staff}
Inderpreet Kaur started as a postdoctoral researcher in the division in March
2020. She is presently working with retrieval of atmospheric ice hydrometeors
using microwave instruments. She shall spend 80\,\% of her time towards the
project. Patrick Eriksson will contribute to all WPs and act as overall project
leader. Luisa Ickes is an Assistant Professor, and has the Arctic region as her
special research interest. She will provide expertise in cloud physics and the
conditions in the Arctic region in general, both when generating training data
and evaluating retrievals (WPs 1, 5 and 6). Leif Eriksson is an Associate
Professor and his research aims at developing methods to observe land and ocean
properties with radar data. In this project, his expertise is sought on the use
of SAR data in the project. He will mainly contribute to WP 1 and WP 7. The
last three persons will spend at least 5\,\% of their time on the project.
During 2022 this will level be 10\,\% for Leif Eriksson. See further Cost
Specification enclosure.



\section{Satellite data}
%
All satellite data of concern in this project are publicly available. SSMIS
data can be obtained at
\url{www.ncei.noaa.gov/access/metadata/landing-page/bin/iso?id=gov.noaa.ncdc:C00810},
and\\ATMS data are available at \url{ftp-npp.bou.class.noaa.gov/}.

{\footnotesize
	\bibliography{j_abbr,refs_pe1,references}
}
\end{document}
